In dieser Phase wurde zuerst das Grundgeruest eines JavaScript Node.js
Webservers kodiert, der als Basis fuer die REST-API dient. Nachdem die
grundsaetzliche Funktionalitaet sichergestellt wurde, wurden die vorgesehenen
URI Resourcen in einzelnen Funktionen des Router-Controllers implementiert.
Anschliessend wurde die Business-Logik, wie Authentifizierung und Authorisierung
implementiert. Daraufhin folgend wurden die Datenmodelle in Code umgesetzt und
abschliessend wurden alle vorgesehenen Tests durchgefuehrt.

\subsection{REST-API}
  \blindtext

\subsection{Datenbank}
  \blindtext

\subsection{Web-Interface}
  \blindtext

\subsection{Tests}
Mit Unit-, Integrations-, und Deploymenttests wird die Wartbarkeit, Performance
und Stabilitaet der Software gewaehrleistet.

  \subsubsection{Unit Tests}
Die Unittests werden mit JavaScript internen Funktionen realisiert und testen
im speziellen die 'User' Datenmodelle und die Authentifizierung.

  \subsubsection{Integrationstests}
Der Integrationstest wurde mit einem Jenkins CI Server vollzogen und verief ohne
Probleme.

  \subsubsection{Deployment}
Test Deployment auf Heroku und mlab.com verliefen ohne Probleme. Die Software
war voll funktionsfaehig. Noetige Installations- und Starthinweise wurden der
Softwaredokumentation, der README Datei, hinzugefuegt.

